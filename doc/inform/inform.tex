\documentclass[12pt]{article}
\usepackage[utf8]{inputenc}
\usepackage{manfnt}
\usepackage{amsfonts}
\usepackage[margin=2cm]{geometry}
\usepackage[spanish]{babel}
\usepackage{graphicx}
\usepackage{fancyhdr}
\usepackage{amsmath}
\usepackage{shortvrb}
\usepackage{amssymb}
\usepackage{amsthm}
\usepackage[T1]{fontenc}

\begin{document}
	\title{Proyecto Sistemas Distribuidos\\Sistema de Almacenamiento Distribuido}
	\author{Jorge J. Morgado Vega\\Roberto García Rodríguez}
	\date{}
	\maketitle

    \section{Introducción}
        Un sistema de almacenamiento distribuido es un sistema de almacenamiento
        que se distribuye en varios nodos, cada uno de los cuales tiene una
        partición de almacenamiento, balanceando la carga de trabajo entre los
        nodos o servidores.

        En este trabajo se propone la idea general de un sistema de almacenamiento
        no centralizado, repartido por la red, cumpliendo las mismas funcionalidades
        que los sistemas estandar.

    \section{Implementación}
        Se desarrolló el proyecto haciendo uso del lenguaje \textbf{Python}
        y la plataforma \textbf{Docker}.

    \subsection{Protocolo CHORD}
        Para la distribución de los recursos en la red se hizo una
        implementación del Protocolo CHORD, el cual brinda mecanismo y funciones
        para la gestión de los recursos en la red, distribuyendo entre los
        servidores del sistema la carga de trabajo.

        Dicho protocolo no cuenta con un mecanismo de replicación de
        recursos, por lo tanto, la caida de un servidor de la red provocaba
        una perdida de los recursos que este poseía.

        Se implementó entonces un modelo de replicación que duplicada los
        recursos en la red, de tal forma (con cierto nivel de tolerancia a
        fallas) que el sistema puede mantener la consistencia y proveer la
        totalidad de los recursos incluso cuando existan servidores fuera
        del sistema.

    \subsection{Conexión}
        Para la conexión con los servidores se utilizó el protocolo TCP/IP,
        usando la biblioteca \emph{rpyc} para los llamados remotos en la red y
        la utilización de los servicios y mecanismo que posee para realizar la
        conexión entre dos puntos de la red.

    \section{API}
        Se provee de una API que sirva de enlace entre el usuario y el sistema,
        la cual se encarga de envía a los servidores los pedidos realizados por
        el usuario. La API del sistema de almacenamiento distribuido contiene
        una serie de funciones que permiten realizar las operaciones de
        almacenamiento y consulta.

    \section{Aplicación}

        Se diseñó un Terminal User Interface en la cual se le brinda al usuario
        una interfaz con la cual puede acceder al sistema y realizar todas las
        operaciones disponibles (subir, descargar, eliminar y listar archivos del
        sistema).
        Debido a la existencia de una API para el sistema, se puede crear
        cualquier tipo de interfaz de usuario sin la necesidad de una implementación
        extra, solo usar los recursos que esta brinda.

\end{document}
